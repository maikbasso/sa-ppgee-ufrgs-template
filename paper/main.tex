\documentclass{ppgeesa}

%\usepackage[latin1]{inputenc}
\usepackage[utf8]{inputenc}
\usepackage{graphicx,amsmath,amssymb,marvosym,latexsym,lineno,hyperref}
\graphicspath{{../images/logos/} {../images/content/}}
\usepackage{adjustbox}

\begin{document}

\title{Título aqui!!!}

\author{
	Maik Basso e Edison Pignaton de Freitas\\
	{Universidade Federal do Rio Grande do Sul (UFRGS) - Av. Osvaldo Aranha, 103 - Porto Alegre - Brasil} \\
	~\\
	\thanks{Maik Basso, maik@maikbasso.com.br, Edison Pignaton de Freitas, edison.pignaton@ufrgs.br}
}

\maketitle
\thispagestyle{empty}\pagestyle{empty}

\begin{abstract} % O abstract não deve exceder 100 palavras
Resumo aqui!
\end{abstract} % O abstract não deve exceder 100 palavras

\begin{IEEEkeywords}
Visão Computacional, VANTs, Sistemas Embarcados, Agricultura de Precisão.
\end{IEEEkeywords}

\section{Introdução}
Técnicas de visão computacional vem sendo amplamente utilizadas para desenvolver sistemas autônomos utilizando VANTs \cite{7500600, 7500671}...

\section{Minha porposta}
O framework ...

\section{Cronograma}
A  Tabela  \ref{tb:cronograma}  apresenta  o  cronograma  com  detalhamento das etapas previstas para a realização deste trabalho.
\begin{table}[!htpb]
\centering
\caption{Cronograma}
\label{tb:cronograma}
\begin{tabular}{|l|c|c|c|c|c|c|c|c|c|}
\hline
\textbf{Etapas} & \multicolumn{2}{l|}{\textbf{2016/1}} & \multicolumn{2}{l|}{\textbf{2016/2}} & \multicolumn{2}{l|}{\textbf{2017/1}} & \multicolumn{2}{l|}{\textbf{2017/2}} & \multicolumn{1}{l|}{\textbf{Fev/18}} \\ \hline
Disciplinas & X & X & X & X &  &  &  &  &  \\ \hline
Proficiência &  &  &  &  &  & X &  &  &  \\ \hline
Estágio Docência &  &  &  &  & X & X &  &  &  \\ \hline
Revisão bibliográfica &  &  & X & X & X & X &  &  &  \\ \hline
Implementação &  &  & X & X & X & X & X & X &  \\ \hline
Testes &  &  & X & X &  &  & X & X &  \\ \hline
Escrita de artigos &  &  & X & X &  &  & X & X  &  \\ \hline
Escrita da dissertação &  &  &  &  & X & X & X & X &  \\ \hline
Defesa &  &  &  &  &  &  &  &  & X \\ \hline
\end{tabular}
\end{table}

\section{Conclusão}
Concluindo aqui...

\bibliographystyle{IEEEtran.bst}
\bibliography{../library/base.bib}

\end{document}

