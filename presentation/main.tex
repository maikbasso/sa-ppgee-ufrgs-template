%definindo as propriedades do beamer
\documentclass[handout]{beamer}
\usetheme{Madrid} % AnnArbor % Antibes % Bergen % Berkeley % Berlin % Boadilla % boxes % CambridgeUS % Copenhagen % Darmstadt % default % Dresden % Frankfurt % Goettingen % Hannover % Ilmenau % JuanLesPins % Luebeck % Madrid % Malmoe % Marburg % Montpellier % PaloAlto % Pittsburgh % Rochester % Singapore % Szeged % Warsaw
\usecolortheme{whale} % albatross % beaver % beetle % crane % default % dolphin % dove % fly % lily % orchid % rose % seagull % seahorse % sidebartab % whale % wolverine	
\useinnertheme{rounded} % rectangles % circles % inmargin % rounded
\useoutertheme{infolines} % infolines % miniframes % shadow % sidebar % smoothbars % smoothtree % split % tree
\usefonttheme{professionalfonts}

% essenciais
\usepackage[utf8]{inputenc}
\usepackage{amsmath}
\usepackage{amsfonts}
\usepackage{amssymb}
\usepackage{textpos}
\usepackage{media9}
\usepackage{multicol}
\usepackage{graphicx}
\graphicspath{{../images/logos/} {../images/content/}}
\usepackage{verbatim}

%remove the icon
\setbeamertemplate{bibliography item}{}

%remove line breaks
\setbeamertemplate{bibliography entry title}{}
\setbeamertemplate{bibliography entry location}{}
\setbeamertemplate{bibliography entry note}{}

% página de título
\titlegraphic{
	\includegraphics[width=2cm]{logo-ufrgs}
	\hspace{0.3cm}
	\includegraphics[width=2cm]{logo-ppgee}
}

% outras páginas
\addtobeamertemplate{frametitle}{}{
	\begin{textblock*}{100mm}(.82\textwidth,-0.9cm)
		\includegraphics[height=0.8cm,width=2.5cm]{logos-top}
	\end{textblock*}
}
%\logo{
	%\includegraphics[width=1cm]{logo-ufrgs}
	%\hspace{0.3cm}
	%\includegraphics[width=1cm]{logo-ppgee}
%}

%defini-se o título
\title[Seminário de Andamento]{Título aqui!!!}

%define-se o nome do autor
\author[Maik Basso]{Maik Basso\\ \textit{maik@maikbasso.com.br} }

%define-se a instituição e o orientador
\institute[UFRGS]{Orientador: Edison Pignaton de Freitas}

%data
\date{2017/2}

\begin{document}

\begin{frame}[plain] %plain remove the bottom bar
	\titlepage
\end{frame}

\begin{frame}{Roteiro da apresentação}
	\tableofcontents
\end{frame}

\section{Introdução}

% \begin{frame}{Introdução}
% 	\begin{columns}
% 		\begin{column}{0.7\textwidth}
%    			\begin{itemize}
%    				\item Veículo aéreo não tripulado (VANT) é toda e qualquer aeronave que não necessita de piloto embarcado para ser guiada
%    				%\item Popularmente conhecidos como DRONES (Dynamic Remotely Operated Navigation Equipment, em português, Equipamento dinâmico de navegação com operação remota)
%    				\item Podem ser autônomos, parcialmente autônomos ou/e controlados remotamente
%    				\item Utilização em aplicações militares, agricultura (AP), lazer e etc.
%    				\item Mapeamento, monitoramento, detecção e etc.
%    			\end{itemize}
% 		\end{column}
% 		\begin{column}{0.3\textwidth}
%     		\begin{center}
%      		\includegraphics[width=0.9\textwidth]{uav-scenario}
%      		\end{center}
% 		\end{column}
% 	\end{columns}
% \end{frame}

% \begin{frame}{Arquitetura estado da arte}
% 		\begin{figure}[ht]
% 		\centering
% 		\includegraphics[scale=.35]{without-framework}
% 		\caption{Arquitetura genérica do estado da arte.}
% 	\end{figure}
% \end{frame}

% \section{Proposta}

% \begin{frame}{Proposta do trabalho}
% 	\begin{center}
% 		\textbf{Framework} Para Controle De Missão E Guiamento Autônomo De Veículos Aéreos Não Tripulados Baseado Em \textbf{Visão Computacional}
% 	\end{center}
% 	\begin{itemize}
% 		\item \textbf{Framework}: 
% 		\begin{itemize}
% 			\item Uma abastração composta de várias bibliotécas e classes de software provendo uma funcionalidade genérica
% 			\item Provê a base para a implementação de uma aplicação
% 			\item Dita o fluxo de controle e dados da aplicação
% 		\end{itemize}
% 		\item \textbf{Visão Computacional}: 
% 		\begin{itemize}
% 			\item Sistemas artificiais capazes de detectar e desenvolver a percepção do meio ambiente através de informações de imagem ou dados multidimensionais
% 			\item Prover a interação do robô com ambiente
% 		\end{itemize}
% 	\end{itemize}
% \end{frame}

% \section{Framework Drone-Control}

% \begin{frame}{Framework Drone-Control}
% 	\begin{block}{Objetivos}
% 		\begin{itemize}
% 			\item Prover a comunicação entre software e hardware;
% 			\item Possibilitar a integração entre softwares desenvolvidos em linguagens diferentes;
% 			\item Manter o desempenho na troca de mensagens (tempo);
% 			\item Ser escalável, possibilitar a comunicação entre $1$ ou $n$ VANTs e $1$ ou $m$ algoritmos clientes;
% 			\item Ser tolerante à falhas (comunicação);
% 		\end{itemize}
% 	\end{block}
% \end{frame}

% \begin{frame}{Framework Drone-Control: arquitetura 1}
% 		\begin{figure}[ht]
% 		\centering
% 		\includegraphics[scale=.35]{FrameworkDroneControl1}
% 		\caption{Arquitetura do framework proposto.}
% 	\end{figure}
% \end{frame}

% \begin{frame}{Framework Drone-Control: arquitetura 2}
% 		\begin{figure}[ht]
% 		\centering
% 		\includegraphics[scale=.35]{FrameworkDroneControl2}
% 		\caption{Arquitetura do framework proposto com sistemas distribuídos.}
% 	\end{figure}
% \end{frame}

% \begin{frame}{Framework Drone-Control: arquitetura 3}
% 		\begin{figure}[ht]
% 		\centering
% 		\includegraphics[scale=.35]{FrameworkDroneControl3}
% 		\caption{Arquitetura do framework proposto com conexão externa.}
% 	\end{figure}
% \end{frame}

% \begin{frame}{Framework: fluxo de execução}
% 	\begin{figure}[ht]
% 		\centering
% 		\includegraphics[scale=.45]{diagrama-drone-server}
% 		\caption{Funcionamento do framework proposto.}
% 	\end{figure}
% \end{frame}

% \begin{frame}[fragile]{Framework: mensagens}
% 	\begin{verbatim}
% {
%     "command": "setPosition",
%     "args": {
%         "x": 10,
%         "y": 5,
%         "z": 2,
%     }
% }
% 	\end{verbatim}
% 	\begin{itemize}
% 		\item Os comandos são abstraídos em mensagens compostas por arrays no formato \textit{JSON} e enviados pelos algorítmos clientes para o framework
% 		\item As mensagens são decodificadas e os comandos tratados pela rotina de execução do framework 
% 	\end{itemize}
% \end{frame}

% \begin{frame}{Teste realizado em simulador}
% 	\begin{figure}[!htb]
%     	\centering
%     	\includegraphics[scale=0.23]{test-drone-control}
%     	\caption{Missão de teste realizado em simulador.}
%     	\label{fig:uav-hardware}
%   	\end{figure}
% \end{frame}

% \begin{frame}{Hardware}
% 	\begin{figure}[!htb]
%     	\centering
%     	\includegraphics[scale=0.33]{uav-hardware}
%     	\caption{Hardware utilizado no projeto.}
%     	\label{fig:uav-hardware}
%   	\end{figure}
% \end{frame}

% \section{Estudo de caso em agricultura de precisão}

% \begin{frame}{Estudo de caso em agricultura de precisão}
% 	\begin{figure}[!htb]
% 		\centering
% 		\includegraphics[scale=.20]{vantpulverizacao}
% 		\caption{VANT realizando a pulverização. Fonte: Google Images.}
% 	\end{figure}
% 	\begin{itemize}
% 		\item Ojetivos:
% 		\begin{itemize}
% 			\item Aplicar de forma pontual e autoregulada maximizando o reservatório
% 			\item Evitar disperdícios
% 			\item Automatizar o processo
% 		\end{itemize}
% 	\end{itemize}
% \end{frame}

% \subsection{Algoritmo para aplicação otimizada de agroquímicos}

% \begin{frame}{Aplicação otimizada de agroquímicos: GNDVI}
% 	\begin{columns}
% 		\begin{column}{0.7\textwidth}
%    			\begin{itemize}
% 				\item Coeficiênte de reflectância:
% 				\[GNDVI = \frac{NIR - G}{NIR + G}\]
% 				\item Aplicando-se na imagem e obtendo GNDVI médio:
% 				\[GNDVI_{AVG} = \frac{\sum_{x=0}^{W} \sum_{y=0}^{H} GNDVI(x,y)}{H \cdot W}\]
% 				\item Resultado por frame:
% 				\[GNDVI_{AVG} = [-1,...,1]\]
% 			\end{itemize}
% 		\end{column}
% 		\begin{column}{0.3\textwidth}
%     		\begin{figure}[ht]
% 				\centering
% 				\includegraphics[scale=40]{rpi-noir-camera}
% 				\caption{Câmera Raspberry Pi NoIR.}
% 			\end{figure}
% 		\end{column}
% 	\end{columns}
% \end{frame}

% \begin{frame}{Aplicação otimizada de agroquímicos: cenário}
% 	\begin{columns}
% 		\begin{column}{0.5\textwidth}
%    			\begin{figure}[ht]
% 				\centering
% 				\includegraphics[scale=1]{fpsxuavspeed}
% 				\caption{Momento da aquisição do frame.}
% 			\end{figure}
% 		\end{column}
% 		\begin{column}{0.5\textwidth}
%     		\begin{figure}[ht]
% 				\centering
% 				\includegraphics[scale=1]{fpsxuavspeed-2}
% 				\caption{Momento da pulverização.}
% 			\end{figure}
% 		\end{column}
% 	\end{columns}
% \end{frame}

% \begin{frame}{Aplicação otimizada de agroquímicos: resultados}
% 	\begin{figure}[ht]
% 		\centering
% 		\includegraphics[scale=.35]{results-gndvi}
% 		\caption{Resultados obtidos nos experimentos de GNDVI.}
% 	\end{figure}
% \end{frame}

% \begin{frame}{Artigo 1}
% 	\begin{figure}[!htb]
%     	\centering
%     	\includegraphics[scale=0.33]{artigo-ndvi}
%     	\caption{Artigo submetido ao Journal Precision Agriculture - Springer.}
%     	\label{fig:artigo-ndvi}
%   	\end{figure}
%   	\begin{itemize}
%   		\item URL: https://link.springer.com/journal/11119
%   		\item Qualis 2016: B1 - Engenharias IV
%   		\item Status atual: Em avaliação
%   	\end{itemize}
% \end{frame}

% \subsection{Algoritmo de guiamento}

% \begin{frame}{Algoritmo de guiamento}
% 	\begin{block}{Objetivos}
% 		\begin{itemize}
% 			\item Realizar a condução autônoma do VANT sobre plantações de agricultura de precisão
% 		\end{itemize}
% 	\end{block}
% 	\begin{exampleblock}{Solução proposta}
% 		\begin{itemize}
% 			\item Utilizar processamento de imagens embarcado para identificação das linhas da plantação e posterior condução do VANT
% 			\item Implementar e melhorar o algoritmo CRD (Crop Row Detection, em português, detecção de linha de plantação) para execução em hardware embarcado de baixo custo
% 		\end{itemize}
% 	\end{exampleblock}
% \end{frame}

% \begin{frame}{Algoritmo de guiamento}
% 	\begin{figure}[!htb]
%     	\centering
%     	\includegraphics[scale=0.28]{pld}
%     	\caption{Algoritmos CRD e de guiamento propostos.}
%     	\label{fig:pld}
%   	\end{figure}
% \end{frame}

% \begin{frame}{Artigo 2}
% 	\begin{itemize}
% 		\item Título: Autonomous UAV Guidance Using Crop Row Detection Algorithm
% 		\item Autores: Maik Basso e Edison Pignaton de Freitas
%   		\item Journal: à definir 
%   		\item Status atual: Escrita em endamento / realizando testes
%   	\end{itemize}
% \end{frame}

% \begin{frame}{Capítulo de Livro}
% 	\begin{itemize}
% 		\item Livro: Imaging And Sensing For Unmanned Aerial Vehicles: Volume 1 - Control And Performance
% 		\item Capítulo: 2. Vision in Indoor and Outdoor Drones
% 		\item Autores: Maik Basso e Edison Pignaton de Freitas
% 		\item Será publicado por: Institution of Engineering and Technology (IET)
%   		\item Status atual: Resumo aprovado, prazo para entrega do capítulo 30/01/2018
%   	\end{itemize}
% \end{frame}


% \section{Conclusões parciais}

% \begin{frame}{Conclusões parciais}
% 	\begin{itemize}
% 		\item O framework proposto tornou-se um facilitador na implementação de algoritmos de visão computacional, tal como sua integração com o hardware
% 		\item O desempenho e flexibilidade do framework estão sendo testados através dos estudos de caso realizados
% 		\item Estudos de caso:
% 		\begin{itemize}
% 			\item Propõe-se um sistema para aplicação pontual e autoregulada de agroquímicos usando VANTs
% 			\item Prova-se a viabilidade de utilização do algoritmo CRD para condução de VANTs sobre plantações de agricultura de precisão
% 		\end{itemize}	
% 	\end{itemize}
% \end{frame}

% \section{Cronograma}

\begin{frame}{Cronograma}
\begin{table}[h]\footnotesize
\centering
\caption{Cronograma}
\label{tb:cronograma}
\begin{tabular}{|l|c|c|c|c|c|c|c|c|c|}
\hline
\textbf{Etapas} & \multicolumn{2}{l|}{\textbf{2016/1}} & \multicolumn{2}{l|}{\textbf{2016/2}} & \multicolumn{2}{l|}{\textbf{2017/1}} & \multicolumn{2}{l|}{\textbf{2017/2}} & \multicolumn{1}{l|}{\textbf{Fev/18}} \\ \hline
Disciplinas & X & X & X & X &  &  &  &  &  \\ \hline
Proficiência &  &  &  &  &  & X &  &  &  \\ \hline
Estágio Docência &  &  &  &  & X & X &  &  &  \\ \hline
Revisão bibliográfica &  &  & X & X & X & X &  &  &  \\ \hline
Implementação &  &  & X & X & X & X & X & X &  \\ \hline
Testes &  &  & X & X &  &  & X & X &  \\ \hline
Escrita de artigos &  &  & X & X &  &  & X & X  &  \\ \hline
Escrita da dissertação &  &  &  &  & X & X & X & X &  \\ \hline
Defesa &  &  &  &  &  &  &  &  & X \\ \hline
\end{tabular}
\end{table}
\end{frame}

\section{Referências}

\begin{frame}[allowframebreaks]{Referências}
	\tiny
	%\cite{woebbecke1995color,jiang2010machine,7003860,7009678,6525881,7500600,7500671}
	\bibliographystyle{IEEEtran.bst}
	\nocite{*}
	%\bibliographystyle{apalike}
	\bibliography{../library/base.bib}
\end{frame}

\begin{frame}{Dúvidas?}
	\begin{figure}[h]
		\includegraphics[width=4cm]{perguntas}
	\end{figure}
	\begin{center}
		\normalsize
		Maik Basso \\
		\textit{maik@maikbasso.com.br}
	\end{center}
\end{frame}

\end{document}